\documentclass[13pt letterpaper]{article}
\usepackage[left=12mm, top=0.1in]{geometry}
\usepackage{graphicx}
\begin{document}
\title{\textbf{Liouville Constant (c)}}
\maketitle
\section{Description}
Liouville number is "almost rational", and can thus be approximated "quite closely" by sequences of rational numbers. Establishing that a given number is a Liouville number provides a useful tool for proving a given number is transcendental. However, not every transcendental number is a Liouville number. No Liouville number can be algebraic.
\section{Relationship}
This calculator consists of the operand and operator functions. Operator functions are further divided into two categories i.e. General Function and Special Function.
General Function include normal multiplication, division, addition and subtraction.
Special Function includes Liouville Constant which is used to either approximate the value or show that number is a transcendental number or not.
\section{UML Diagaram}
\includegraphics[width=\linewidth]{UML}
\end{document}