\documentclass[13pt letterpaper]{article}
\usepackage[left=12mm, top=0.1in]{geometry}
\begin{document}
\title{\textbf{Interview Question}}
\maketitle
\begin{enumerate}
\item What is your name?\\
\textbf{Ans-} My name is Subhankar Joshi.
\item What is your domain of study?\\
\textbf{Ans-} I am doing a master's in applied mathematics from Delhi university.
\item What is/are the most common number/numbers you generally use?\\
\textbf{Ans-} If I talk about general purpose use  I found out that Natural Logarithm of 2 (lne(2)), Pi($\pi$) used in computer concepts, population concepts, physical phenomenons and Golden Ratio($\Psi$) used in architecture, art, and design, etc are the most commonly used Eternity numbers. From the mathematical point of view, Euler’s Number (e) is the most common Eternity number used in mathematical equations, Compound Intrest, etc. So a set of commonly used Eternity numbers are Pi($\pi$), Golden Ratio($\Psi$) and Natural Logarithm of 2, Euler’s Number (e).
\item What is/are the most uncommon number/numbers you generally use?
\textbf{Ans-} Champernowne Constant (C$_{10}$), Universal Parabolic Constant (P$_{2}$), Liouville Constant (c) and are some of the few Eternity numbers that I rarely come across in my field of study. 
\item What is the application domain of Liouville constant(c)?\\
\textbf{Ans-} Liouville constant(c) is used in algebraic numbers and approximations of transcendental numbers. In general it has very limited use.
\item Is there any Eternity number that according to you can prove a game-changer for the calculator?\\
\textbf{Ans-} Yes according to me, Gaussian Integral ($\sqrt{\pi}$) is the game-changer for the calculator because it is widely used in physics especially in quantum physics and mathematics too. People from a physics background can also get benefitted by this.
\item Is there any suggestion that you can provide to improve the overall quality of the calculator?\\ 
\textbf{Ans-} Yes, all the commonly used Eternity numbers should be placed in the calculator in such a way that the user finds it easy to locate and use them. If the calculator can record previous calculations then the overall quality of the calculator will increase significantly. 
\end{enumerate}
\end{document}